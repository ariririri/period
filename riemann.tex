\chapter{リーマン面の理論}
\section{周期積分、ヤコビ多様体}
実際にリーマン面の理論の後半を読み,周期の関連を見る.

\subsection{位相幾何からの準備}
議論をする上で最低限のホモトピー論,ホモロジー論を定義する.
\begin{screen}
\begin{dfn}
 $X$上の2つの連続なpath $f,g: [0,1] \to X$に対し,連続写像$H:[0,1] \times [0,1] \to X$で
 $H(0, \cdot) = f(\cdot), H(1, \cdot)$となるものが存在するす時,$f,g$は\textbf{ホモトピック}という.
\end{dfn}
\end{screen}

\begin{screen}
\begin{dfn}
 $X$と$x \in X$に対して基点,つまり始点と終点が$x$となる連続なpath全体を$F$と表す.$\alpha, \beta \in F$に対して,
 \begin{equation*}
 \alpha \cdot \beta (t) = \begin{cases}
    \alpha(2t) & ( t \le 1/2) \\
    \beta(2t-1) & (t \ge 1/2)
  \end{cases}
 \end{equation*}
とすると,これは積構造を定める.
これをホモトピーによる同値関係で割ったものを\textbf{基本群}といい$\pi(X,x)$で表す.
すぐ計算すればわかるように,基本群上でも上で定めた積構造がwell-definedであり,この演算について群になる.
また,弧状連結である等,基点のとり方によらない場合は$\pi(X)$と表すこともある.
\end{dfn}
\end{screen}

\begin{screen}
\begin{dfn}
 $A \subset X$に対し,$F(x, 0) = x , F(x, 1) \in A , F(a, 1) = a$となる写像$F$を変位レトラクトという.
これは,$X$上の恒等写像と$A$へのレトラクション($A$への制限が恒等写像になるもの)の間のホモトピーの存在を意味する.
この時,$id$とレトラクトはホモトピー同値になる.ただし$A$に写像が制限されていないので,ホモトピー同値はホモトピックと同値になる.
\end{dfn}
\end{screen}

ホモロジー・コホモロジーについて定義する.
具体的な話を書きすぎると死んでしまうが,

\begin{itemize}
\item 特異ホモロジー
\item 単体ホモロジー
\item 特異コホモロジー
\item Kronecker積
\end{itemize}
について説明する.

$\Delta^n := \{ x \in \mathbb{R}^{n+1} \mid  \sum x_i =1, \forall x_i \ge 0 \}$とする.
$S_n(X) := \{f: \Delta^n \to X \mid f \mbox{は連続写像}\}$とし,
$C_n(X) := \oplus_{f \in S_n(X)} \mathbb{Z}$とする.
$i_j: \Delta_n \to \Delta_{n+1}$を第$j$成分を0にし,それ以降は一つずらしとする写像とする.
$d: S_n(X) \to S_{n-1}(X), \sum_{j} f \mapsto (-1)^j f \circ i_j$とする.
すると$d_n \circ d_{n+1} = 0$となり,

\begin{screen}
\begin{dfn}
$\mathcal{U} =  \{U_i\}_{i \in I}$を$X$の開被覆とする. 任意の有限個の$i_1, \ldots ,i_k \in I$に対し$ \cap U_i$が可縮な時,$\mathcal{U}$を\textbf{単純被覆}という.
$X$が$n$次元多様体であって,
\end{dfn}
\end{screen}

\subsection{ポアンカレの補題とドラムの定理}
