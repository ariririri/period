\chapter{GAGA}
射影代数多様上の層とそのanalyticなものが定めるコホモロジーの一致や圏同値を示す.
最初に代数多様体を定義し,次にanalytic spaceを定める.
代数多様体の複素化を定義し,射影代数多様体の場合の関係について示す.


\section{Analytic Spaces}
最初にローカルな世界でanalyticを定義する.
\begin{screen}
\begin{dfn}
$U \subset \mathbb{C}^n$がanalyticとは$\forall x \in U$に対し,ある$x$を含む$W \subset \mathbb{C}^n$上の正則関数$f_1, \ldots ,f_k$が存在し,
\begin{equation*}
  U \cap W = \{x \in W \mid f_1(x) = \ldots  f_k(x) = 0 \}
\end{equation*}
となること.
\end{dfn}
\end{screen}

こは複素多様体とは限らない.
$U \cap W$上で$ \frac{\partial f_i}{\partial x_j}$のrankが$U$のとり方によらない時,複素多様体の閉部分多様体となる.

$U$の位相を調べる.
この時$U$はlocally closedとなりまたlocally compactとなる.
\begin{screen}
\begin{dfn}
位相空間$X$の部分集合$A$がlocally closedとは,以下の同値な条件を満たすことである.
\begin{itemize}
  \item あるopen set $U$と closed set $V$が存在し,$A = U \cap F$となること.
  \item $x \in A$に対し,ある$X$の近傍$W$が存在し,$A \cap W$が$W$上closedであること
\end{itemize}
\end{dfn}
\end{screen}

これがら同値なことは少し議論すればよい。

GAGAでは層を以下で定義している.
\begin{itemize}
  \item 位相空間$X$
  \item $x \in X$に対する関数$\mathcal{F}_x$
  \item $\mathcal{F}$を$\mathcal{F}_x$のdisjoin unionとする(位相は何で定める?)
\end{itemize}
$\pi: \mathcal{F} \to X$を$f \in \mathcal{F}_x$の時$\pi(f)=x$とする.
さらに
\begin{itemize}
  \item $f \in \mathcal{F}$に対し$f \in \mathcal{F}$の近傍と$x$の近傍が同相であるとする.
  \item $f,g$に対し, $-f , f + g$がcontinuos(加法は$\pi(f) = \pi(g))$の時のみ.
\end{itemize}

これは開集合$U$に対し,そのsectionを対応させることにより,以前定義した層の定義と一致する(はず)


\begin{rem}[疑問]
$\mathcal{F}$の位相はどうやって定める
\end{rem}

$X$上$\mathbb{C}$への連続関数が作る層を$\mathcal{C}(X)$とする.
$\mathcal{C}(\mathbb{C}^n)$を$\mathbb{C}$に値を取る連続関数のなす層とし,$\mathcal{H}$を$C^n$上の正則関数のなす層とする.



\begin{rem}[疑問]
$U \subset \mathbb{C}^n$上の正則関数のなす層を$\mathbb{C}^n$の連続関数のなす層から$U$上の連続関数のなす層への写像を使い定義している。
これは通常定義される正則関数のなす層を一致する.
\end{rem}

\begin{screen}
\begin{dfn}
 $U,V$をanalytic spaceとする.$\phi: U \to V$が\textbf{holomoprhic}とは以下を満たすことである.
 \begin{itemize}
   \item $\phi$は連続
   \item $\mathcal{H}_{\phi(x), V} \to \mathcal{H}_{x, U}, f \mapsto f \circ \phi$がwell-definedであること.
         つまり$f \circ \phi \in \mathcal{H}_{x, U}$となること.
 \end{itemize}
\end{dfn}
\end{screen}

\begin{rem}
この定義は$U \to V$が通常の意味で正則であることと同値になるか?
また,これは環つき空間としての射となっていることを意味する.
特にlocal性は$m_{\phi(x)} = \{f \mid f(\phi(x))\}$より, これは $\phi*(m_{\phi(x)}) \subset m_{x}$となるので問題ない.
\end{rem}

analytic subsetやholomorphicは直積で保たれる.
つまり$U, V$がanalyticなら$U \times V$はanalyticで$\phi, \phi'$がholomorphic$\phi \times \phi' : U \times U' \to V \times V'$はholomorphic.


\section{The analytic space associated to an algebraci variety}
最初に$\mathbb{C}$上のalgebraic varietyとregular mapを定義する.

\begin{screen}
\begin{dfn}
有限生成$\mathbb{C}$代数$R$に対し,$(\mathrm{Spm}R, R)$を\textbf{アフィン代数多様体}といい、
2つのアフィン代数多様体$(\mathrm{Spm}R, R)$と$(\mathrm{Spm}S, S)$に対して$\mathbb{C}$準同型写像$\psi: S \to R$と$\psi$から定まる写像$\psi^a: \mathrm{Spm}R \to \mathrm{Spm}S$が存在する時,
$(\psi, \psi^a)$を(正則な)射という.
\end{dfn}
\end{screen}
$I$を含む極大イデアル全体の集合を$V(I)$とすると,$V(I)$全体を閉集合とする位相が定まり,Zariski位相という.

\begin{rem}
  スキームと同様に$f \in R$に対し,$D_m(f):= R_f$とする層が定まりこれにより代数多様体も局所環つき空間として考えられる.
\end{rem}

これに付随するanalytic spaceが定義できる.
代数多様体はlocalにはAffine代数多様体であり,その時$\mathrm{Spm}R$は$\mathbb{C}^n$のZariski閉集合に対応する.
よってこれらにはanalytic spaceの構造が定まり,実は全体ではりあわせることができる.

\begin{prop}
There exists on $X$ a unique structure of an analytic space such
 that, for every chart $\phi : V \to U$, the Z-open set $V$ is open, and $\phi$ is an analytic
 isomorphism of V (equipped with the analytic structure induced by that of $X$)
 onto $U$ (equipped with the analytic structure defined in $n^o 1$).
\end{prop}

\section{Thre correspondence between algebraic sheaves and coherent analytic sheaves}

\section{The analytic sheaf associated to an albgeraic sheaf}
$X$を$\mathbb{C}$上のalgberaic varietyとする.
それに対し,$X^h$を$X$に付随するanalytic spaceとする.

$O_x$加群の層ををalgebraic sheafといい,$M$をanalytic spaceとした時,$O_{M}$加群の層をanalytic sheafという.


\section{Projective varieties. Statements of the theorems.}
GAGAのmain theoremについて説明する

$X$をprojective variety,ここでは$\mathbb{P}^r(\mathbb{C})$の閉部分多様体する.

\begin{thm}
$X$上のcoherent algeberaic sheaf $\mathcal{F}$に対し,
\begin{equation*}
 \epsilon: H^q(X, \mathcal{F}) \to H^q(X^h, \mathcal{F}^h)
\end{equation*}
はisomorphism
\end{thm}

\begin{thm}
$\mathcal{F}, \mathcal{G}$を$X$上のcoherent algebraic shaefとする.この時,
\begin{equation*}
 \mathrm{Hom}(\mathcal{F}, \mathcal{G}) \sim \mathrm{Hom}(\mathcal{F}^h, \mathcal{G}^h)
\end{equation*}
\end{thm}

\begin{thm}
$X^h$上のcoherent analytic sheaf $\mathcal{M}$に対し,$X$上のalgberaic sheaf $\mathcal{F}$で$\mathcal{F}^h = \mathcal{M}$となるものが存在する
\end{thm}

これを一つずつ示す.

\section{Proof of theorem 1}
証明の前にいくつか補題を示す.

\begin{lem}
$X$上の層$\mathcal{F}$に対し,$X \subset Y$のとき,
$Y$のopn set $U$に対し,$\mathcal{F}^e(U)$を
\begin{equation*}
  U \mapsto \begin{cases}
    \mathcal{F}(U) & (if U \subset X) \\
    0 & (otherwise)
\end{cases}
\end{equation*}
で定める前層の層化とする.
この時,
\begin{equation*}
\mathcal{F}^e_x =
\begin{cases}
    \mathcal{F}_x & (if x \in X) \\
    0 & (otherwise)
\end{cases}
\end{equation*}
となる.
\end{lem}
\begin{proof}
前層のとり方から明らか.
\end{proof}

\begin{lem}
この時,cohomologyは一致する.
つまり,$X \subset Y$の時
\begin{equation*}
 H^q(X, \mathcal{F}) = H^q(Y, \mathcal{F}^e)
\end{equation*}
となる.
\end{lem}
\begin{proof}
check cohomologyで考えるとよい.
$Y$の任意のopen coverに対し,
Fineな十分小さいOpen Cover$\mathcal{U} = \{U_i\}$を取ると
$U_i \subset X$以外では$\mathcal{F}(U_i) = 0$としてよい.
よって十分細かいopen coverに対し,
$H(\{U_i \cap X\}, \mathcal{F}) = H(\mathcal{U}, Y)$となる.
これより,inductive limitwをとっても一致する.
\end{proof}

$X$でのコホモロジーと射影空間のコホモロジーが一致するので,射影空間で考える.
射影空間の場合に
構造層のコホモロジーが$0$を除き消えること
analyticもdolebuの定理から消えることを示す.
$O(n)$の場合は完全列を誘導し,inductionから示す.
一般の場合はcoherent sheafであるため,$O(n)$の直積から全射があり,five term lemmaで示す.

\section{Proof of theorem 2}
層のflat性とテンソル積から言えそう.
