\section{Projective Scheme}

Projcetive Schemeの定義をする.
GAGAの対象となる射影代数多様体は$\mathbb{P}^n_{\mathbb{C}}$のclosed subschemeである.

\begin{dfn}
$B$が次数つき$A$代数であるとは
$B = \oplus B_n$とかけ,$B_n B_m \subset B_{n+m}$となること
\end{dfn}



\begin{lem}
  $f \in B_+$をdegree $r$のhomogeneousな元とする.
\begin{enumerate}
  \item canonicalな射は$\theta: D_+(f) \to \mathrm{Spec}B_{(f)}$同相となる
  \item $D_+(g) \subset D_+(f)$の時,$\alpha = g^r f^{- \mathrm{deg} g}$とする.
  この時$\theta(D_+(g)) = D(\alpha)$
  \item 自然な射$B_{(f)} \to B_{(g)}$は$(B_{(f)})_{\alpha} = B_{(g)}$をinduceする.
  \item その他いろいろ
\end{enumerate}
\end{lem}
1を示す.
方針は以下の通り.
\begin{itemize}
  \item $\mathrm{Proj}B$が$\mathrm{Spec}B$の部分位相であることを示す.
  \item 環の間の射を用いて自然な射を誘導し,連続であることを示す
  \item 射が全射であることを示す
  \item 射が単射であることを示す.
  \item 開写像であることを示す.
\end{itemize}

\begin{screen}
\begin{dfn}
  $X_i:=\mathrm{Spec}A[T_iT_j^{-1}]$とすると$X_{ij} = X_{ji}$となるので,これを張り合わせたものを$A$のProjective Spaceといい$\mathbb{P^n}_A$で表す.
  Projective Spaceのclosed subschemeをProjective Schemeという.
\end{dfn}
\end{screen}
