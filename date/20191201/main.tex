\chapter{第一回周期セミナー@12/01}
12月1日に行われた周期セミナーの初回で説明である.
第一回では層の定義と層のコホモロジーについて議論した.

周期の数論的な関係を見るために,代数的なDeRhamの定理を\url{https://arxiv.org/pdf/1302.5834.pdf}に基づきまとめる.

ここでは層と層係数のコホモロジーを使ってドラムの定理を計算する.
そこで最初に層の準備、及びスペクトル系列についての内容を記載する.

\begin{itemize}
  \item 微分形式のなす層を定義し,それがfineであることを示す.
  \item パラコンパクト位相空間上のfine sheafはacyclic
  \item 二重複体のコホモロジーからacyclic resolutionの時のコホモロジーの同型を示す.
  \item コホモロジーの同型とPoincareの補題から可微分多様体と複素多様体の場合のドラムの定理を示す.
  \item GAGAと複素多様体の時のドラムの定理から射影代数多様体の場合にドラムの定理を示す.
  \item チェックコホモロジーを使って代数多様体のドラムの定理を広い範囲に拡張している.
\end{itemize}

コホモロジーのNotaiton
\begin{itemize}
  \item $\mathcal{H}^k(A())$: 層のcomplexのk次のImage/Ker
  \item $\mathfrak{h}^k(T(L))$: 層のcomplex$L$をアーベル群の完全関手で飛ばした先のコホモロジー.
  \item $H^K(U,L)$: 層$L$のcomplexの$U$でのセクションのコホモロジー.
  \item $\mathbb{H}^k(X, L)$: 二重複体(今回は$L$のGodement resolution)に対するTotalのコホモロジー
\end{itemize}
余力があればやりたいこと
\begin{itemize}
  \item $M$の特異コホモロジーと定数層のコホモロジーの(大域切断)一致
\end{itemize}

\begin{thm}
\begin{equation*}
H^n(X,\Omega_{alg}) = H^n(X_{an},\mathbb{C})
\end{equation*}
\end{thm}
\section{層の議論}
ここでは層と層の間の射について定義する.
\begin{itemize}
\item 層の層化側の定義
\item 層の貼り合わせの条件
\item 環つき空間の射
\item $f:(X, O_X) \to (Y, O_Y)$から作れる層.
\end{itemize}


\subsection{層の定義}

\begin{screen}
\begin{dfn}
 任意の開集合$U \subset X$に対し,$\mathcal{F}(U)$が対応し,
 $v \subset U$なら制限写像$|_V: \mathcal{F}(U) \to \mathcal{F}(V)$が存在し、以下を満たす時
 $\mathcal{F}$を前層という.
 \begin{itemize}
   \item $F(\emptyset) = 0$
   \item $|_U:F(U) \to F(U) = id$
   \item $|_W \circ |_V = |_W$
 \end{itemize}
 前層$\mathcal{F}$が任意の開集合$U$とその開被覆$\{U_i\}$に対し,以下を満たす時層という.
 \begin{itemize}
   \item $s \in F(U)$に対し,$s|_{U_i}=0$の時,$s=0$.
   \item $s_i \in F(U_i)$に対し,$s_i|_{U_{ij}}= s_j|_{U_{ij}}$となる時,$s \in U$で$s|_{U_i} = s_i$となる.
 \end{itemize}
\end{dfn}
\end{screen}


\begin{epl}
位相空間$\mathbb{C}$を考え,その開集合$U$に対し,
\begin{equation*}
F(U) := \{f:U \to \mathbb{C} \mid f \mbox{is continuous} \}
\end{equation*}
とする.$|_U$を定義域の制限で定める.この時$F$は層になる.
\end{epl}

\begin{lem}
前層$\mathcal{F}$が層であることは
任意の開集合$U$とその開被覆$\mathcal{U}$に対し,
\begin{equation*}
 0 \to \mathcal{F}(U) \to \prod_{U_{\alpha} \in \mathcal{U}} \mathcal{F}(U_{\alpha})
   \to \prod F(U_{\alpha} \cup U_{\beta})
\end{equation*}
がexact.
\end{lem}
\begin{proof}
自明.ちゃんと証明を書く場合はまた今度.
単射性は左側の完全性
張り合わせは右側の全射性に対応
\end{proof}

前層$\mathcal{F}$に対し,
\begin{equation*}
    \mathcal{F}_x := \varinjlim_{x \in U}\mathcal{F}(U)
\end{equation*}
これを前層の茎(stalk)という.

前層の層化を定義する
\begin{screen}
\begin{dfn}[層化]
$X$上の前層$\mathcal{F}$に対し,
\begin{equation*}
\mathcal{G}(V)= \{ (s_x) \in \coprod_{x \in V} \mathcal{G}_x \mid \forall x, \exists U_x ,f  \in \mathcal{F}(U_x) \mbox{s.t. } \forall y \in U_x, s_y = f_y \}
\end{equation*}
で定めた層$\mathcal{G}$を層化という.
\end{dfn}
\end{screen}


\begin{lem}
上で定めた$\mathcal{G}$は層になる.
\end{lem}
\begin{proof}
単射性を示す.
開被覆$\{\mathcal{U_i}\}$に対し,$s|_{U_i} = 0$なら
$s = (s_x)$とした時にすべての$s_x$に対し$0$となるので$s = 0$となる.
張り合わせは$s_i|_j = s_j|_i$の時,$s = (s_x)$を$x \in U_i$に対しては$s_x =s_i|_x$とする.
この時,$x \in U_{ij}$に対し,$s_i|_x = s_j|_x$となるので,$s$はwell-definedであり,$s|_{U_i} = s_i$となる.
\end{proof}

\begin{lem}
自然な射$f: \mathcal{F} \to \mathcal{G}$とすると,$\forall x \in X$に対し,
$\mathcal{F}_x \simeq \mathcal{G}_x$を誘導する.
\end{lem}


\begin{screen}
\begin{dfn}
$X$のOpen Base $\mathcal{B}$に対して層の性質を満たすものを,$\mathcal{B}$-sheafという.
\end{dfn}
\end{screen}

\begin{lem}
$\mathcal{B}$-shaef $\mathcal{F}$に対し,$X$上の層$\mathcal{G}$で$\forall U \in \mathcal{B}$に対し,$\mathcal{F}(U) = \mathcal{G}(U)$となるものが存在する.
\end{lem}
層化同様の方法で
\begin{equation*}
\mathcal{G}(V)= \{ (s_x) \in \prod \mathcal{G}_x \mid \forall x, \exists U_x ,f  \in \mathcal{F}(U_x) \mbox{s.t. } \forall y \in U_x, s_y = f_y \}
\end{equation*}
とすればこれは層になる.
$\mathcal{F}(U) = \mathcal{G}(U)$はinductive limitのuniversalityからもわかるし,
層化の構成で作ったものとの同型を層の性質からも直接示せる.
$\mathcal{F}(U) \\mathcal{G}(U)$は$\{U_x \}$を取ることにより,開被覆となり,全射性がわかる.また単射性も層の性質からわかる.

層の定義を満たす具体例を紹介する.
層の典型的な例は位相空間$X$に対し,$F(U)$を$U$から$\mathbb{R}$への連続関数全体です.


\begin{rem}
 上を用いるとある開被覆$\{U_i\}$と$U_i$上の層$\mathcal{F}_i$が存在し,$\mathcal{F}_i|_j = \mathcal{F}_j|_i$ならば
 $X$上の層$\mathcal{F}$で$F|_i = \mathcal{F}_i$となるものが存在する
 ($\{U_i\}$とその開部分集合はopen baseになるので)
 なので,スキーム同士を張り合わせる場合は上の条件を満たせば問題ない.
\end{rem}

\subsection{層の射}
$X$上の層の射の定義と射の全単射を定義する
\begin{screen}
\begin{dfn}
$f: \mathcal{F} \to \mathcal{G}$とは

\begin{tikzcd}
  \mathcal{F}(U) \ar[r, "f_U"] \arrow[d, "|_V"'] & \mathcal{G}(U) \ar[d, "|_V"] \\
  \mathcal{F}(V) \ar[r, "f_V"] & \mathcal{G}(V)
\end{tikzcd}

が可換となること
\end{dfn}
\end{screen}

開被覆で定まる層があれば,それをもとに全体に張り合わせられる.

\begin{screen}
\begin{dfn}
$X$上の層$f: \mathcal{F} \to \mathcal{G}$に対し,$\mathrm{Ker}f$を
\begin{equation*}
    U \to \mathrm{Ker} (f(U))
\end{equation*}
で定める層とし,$\mathrm{Im}f$を
\begin{equation*}
    U \to \mathrm{Im} (f(U))
\end{equation*}
の層化で定義する.

\end{dfn}
\end{screen}

\subsection{層の射が誘導する層}
位相空間の間の連続写像が誘導する層
$f: X \to Y$に対し
$X$上の層$\mathcal{F}$に対し,

層の延長

\subsection{環つき空間と射}

環つき空間

環つき空間の間の射
\begin{screen}
\begin{dfn}
環$R$に対し,素イデアル全体のなす集合$\mathrm{Spec}R$に$D(f)$を開基とする位相をいれる.
\end{dfn}
\end{screen}

スキームや代数多様体が環つき空間になっていること(これはRemark)



\input{date/20191201/cohomologyofsheaf}